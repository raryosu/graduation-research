\section{おわりに}

本研究では,いくつかの情報量規準を分割停止規準として採用してX-meansでクラスタ数推定を行った.
その結果,2次元空間における混合等方Gauss分布のクラスタリングにおいてはBICやcAICが,
3次元空間における混合等方Gauss分布のクラスタリングにおいてはBICが適していることがわかった.
2次元空間においてはAICを採用した場合,クラスタ数を過大に見積もってしまう問題が見受けられた.

以上より,等方Gauss分布により生成されたデータのクラスタリングを行う際は
BICを分割停止規準として採用することで適切なクラスタリングを行うことができると
言える.

今後は,AIC,cAIC,BIC以外の情報量規準を用いたクラスタリングや,
クラスタリング対象のデータを変更するなどして,それぞれのデータのクラスタリングに
最も適した情報量規準を探求していきたい.

