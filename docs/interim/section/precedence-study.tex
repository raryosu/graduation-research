\section{先行研究}

\subsection{クラスタリング}
クラスタリングとはデータを教師なし学習により任意の数のクラスタに分ける手法である.
多くのクラスタリング手法においては,データの類似度をユークリッド距離やマンハッタン距離などの
距離尺度によって定義し,それによってクラスタを抽出する.
クラスタリングはデータ解析,データマイニング,パターン認識など様々な分野で用いられる.

\subsection{K-means}
K-means$^{1)}$は,多次元空間上のデータ点集合について,各データが属するクラスタを同定する最もよく使われるクラスタリング手法の一つである.
K-meansは,以下の2つの手順を繰り返すことでクラスタリングを行う.
\begin{enumerate}
  \item 各データ点とデータ点の距離を求め,各データ点を最も近いセントロイドのクラスタに割り当てる.
  \item クラスタに所属するデータの平均を新たなセントロイドとする.
\end{enumerate}
セントロイドが移動しなくなったらクラスタリングを終了する.

% \begin{algorithm}
%   \caption{K-means Algorithm}
%   \label{alg:k-means}
%   \begin{algorithmic}
%     \REQUIRE
%       $N$: データ数, $K$: クラスタ数, $\lambda$: 収束条件,
%       ${\bm X} = ({\bm x_0}, {\bm x_1}, \cdots, {\bm x_N})$
% 
%     $K$個のセントロイドとしてランダムに配置する.
%   \end{algorithmic}
% \end{algorithm}
% 

K-meansによるクラスタリングは,事前にクラスタ数を指定することにより,クラスタリングが行われる.
したがって,クラスタ数が未知の場合,K-meansを用いることはできない.

