\documentclass[10.5pt,a4j]{ltjsarticle}

% 卒業研究報告書スタイルファイル
\usepackage{layouts/tsuyama/eeses}

\usepackage[tocgraduated]{tocstyle}
% ソースコード表示
\usepackage{listings}
% 色
\usepackage{xcolor}
% 数学関連
\usepackage{amsmath, amssymb}
% リスト制御
\usepackage{enumitem}
% 画像
\usepackage{graphicx}
% shaded環境の背景色の定義
\definecolor{shadecolor}{gray}{0.80}
% 枠
\usepackage{ascmac}
\usepackage{tcolorbox}
% url表記
\usepackage{url}
% ハイパーリンク
\usepackage[pdfencoding=auto]{hyperref}
% フォント
\usepackage{layouts/lualatexsets/fonts}
% Tikz関係
\usepackage{tikz}
% 証明などのスタイル
\usepackage{layouts/others/theorem}
% セクションの表示スタイル
%\usepackage{layouts/others/section}
% ベクトル表記
\usepackage{bm}
\def\vector#1{\boldsymbol{#1}}
% 疑似コード
% \usepackage{algorithm}
% \usepackage{algorithmic}
% 画像・図表等のrefコマンド
\def\thmref#1{Thm. \ref{#1}}
\def\lmmref#1{Lemma. \ref{#1}}
\def\figref#1{図\ref{#1}}
\def\eqref#1{(\ref{#1})式}
\def\tableref#1{表\ref{#1}}

%%% ドキュメント情報 %%%
% 著者
\author{萩原 涼介}
% タイトル
\title{クラスタ数推定に用いる最適な情報量基準の探求}
% 指導教員
\adviser{藤田 一寿}
% 日付
\date{平成29年10月10日}
% 学科
\affiliation{情報}
% 種別
\kind{\midpre}
% pdfファイル情報
\hypersetup{%
  colorlinks=true,%
  urlcolor=black,%
  linkcolor=black,%
  citecolor=black,%
  linktocpage=true,%
  bookmarks=false,%
  pdftitle={卒業研究 中間報告書},%
  pdfsubject={クラスタ数推定に用いる最適な情報量基準の探求},%
  pdfauthor={Hagihara Ryosuke},%
  pdfkeywords={クラスタリング; 情報量規準; クラスタ数推定; 機械学習}
}


\begin{document}
\maketitle

% Abstract
\begin{abstract}
Clustering is a one of methods to divide data into several groups called cluster.
It is used in various fields such as data analysis, data mining, image processing, and pattern recognition.
In many clustering methods including $k$-means are assumed that the number of clusters is known in advance and clustering is performed by specifying the number of clusters.
In spite of it, in general, there are few data whose number of cluster is known in advance.
Even when the number of clusters is unknown, it is important to estimate the number of clusters appropriately.
X-means is one way to estimate the number of clusters. 
It is to suppose to the data generated from mixed isotropic Gaussian distribution, and estimates the number of clusters by estimating the parameters of the probability distribution.
It uses an estimator called information criterion.
Many of information criterions are proposed by resarchers like AIC (Akaike Information Criterion) and BIC (Bayesian Information Criterion).
In to estimate the number of clusters, however, it is not known which information criterion is valid for what kind of data.
In this study, the optimal information criterion used for estimating the number of clusters is clarified through quantitative experiments.
\end{abstract}


% ToC
\tableofcontents

% はじめに
\section{緒言}
クラスタリングとはデータを教師なし学習により任意の数のクラスタ(データのグループ)に分ける手法である.
クラスタリングはデータ解析,データマイニング,画像処理,パターン認識など様々な分野で用いられる.
$k$-meansを始めとする多くのクラスタリング手法では,予めクラスタ数がわかっているものとして,
クラスタ数を指定しクラスタリングを行う.
しかし,データに対し最適なクラスタ数を指定しなければ,最適なクラスタリング結果を得ることはできない.
それにも関わらず,一般にクラスタ数が事前にわかっているデータは少ない.
その為,クラスタ数が未知である場合に対しても,適切にクラスタ数を推定しクラスタリングを行うことは重要な課題となっている.

クラスタ数推定を行う手法の一つにX-meansがある.
X-meansは,データが混合等方Gauss分布から生成されたと想定して,
その確率分布のパラメータを推定することにより,クラスタ数推定を行う.
X-meansは情報量規準とよばれる,確率分布とデータの分布の当てはまり具合(モデルの良さ)を表す指標を用い,クラスタ数推定を行う.
その情報量基準は多くの研究者により様々なものが提案されている.
代表的なものに,1973年に赤池が提案したAIC (Akaike Information Criterion) や,
Bayesの定理によって算出される事後確率を用いるBIC (Bayesian Information Criterion) ,
AICの課題を解決したcAIC (Conditional Akaike Information Criterion) などがある.
一般にX-meansは,クラスタ数を決定する上で BIC を用いている.
しかし,クラスタ数推定において,どの情報量規準がどのようなデータに対し有効かは分かっていない.
そこで本研究では,クラスタ数推定に用いる情報量規準として最適なものを数値実験を通し明らかにする.

本研究では,AIC, cAIC, BICと呼ばれる情報量規準をそれぞれ用いたX-meansにより混合等方Gauss分布から生成されたデータ
および実データのクラスタ数推定およびクラスタリングを行った.
そして,どの情報量基準がクラスタ数推定に有効かを調査した.

本報告書の構成は次のとおりである.
第2章では,既存のクラスタリング手法である$k$-meansおよびMean shiftのアルゴリズムの紹介を行う.
第3章では,本研究で利用するX-meansの理論の説明を行う.
まず,モデルと真の確率分布との近さを計る指標である情報量基準の例としてKullback-Leibler情報量について述べ,
それと最尤推定との情報量規準の関係性について詳しく述べる.
その後,X-meansの手法について述べる.
第4章では,本研究により得られた実験結果について述べる.
第5章では,本研究を通してのまとめおよび今後の課題について述べる.

% 先行研究
\section{先行研究}

\subsection{クラスタリング}
クラスタリングとはデータを教師なし学習により任意の数のクラスタに分ける手法である.
多くのクラスタリング手法においては,データの類似度をユークリッド距離やマンハッタン距離などの
距離尺度によって定義し,それによってクラスタを抽出する.
クラスタリングはデータ解析,データマイニング,パターン認識など様々な分野で用いられる.

\subsection{K-means}
K-means$^{1)}$は,多次元空間上のデータ点集合について,各データが属するクラスタを同定する最もよく使われるクラスタリング手法の一つである.
% K-meansは,以下の2つの手順を繰り返すことでクラスタリングを行う.
% \begin{enumerate}
%   \item 各データ点とデータ点の距離を求め,各データ点を最も近いセントロイドのクラスタに割り当てる.
%   \item クラスタに所属するデータの平均を新たなセントロイドとする.
% \end{enumerate}
% セントロイドが移動しなくなったらクラスタリングを終了する.

$D$次元空間上の確率変数$x$の$N$個の観測点で構成されるデータ集合
$\{\vector{x}_1, \vector{x}_2, \cdots, \vector{x}_N\}$があると仮定する.
このデータ集合を$K$個のクラスタに分割することを考える.
直感的には,クラスタとは,その内部のデータ点間の距離が外部のデータとの距離と比較して小さいテータのグループであるとみなせる.
ここで,セントロイド$\vector{\mu}_k$を導入する.
K-meansはベクトルの集合$\{\vector{\mu}_k\}$だけでなく,全データ点をうまくクラスタに対応させ,
各データ点から $\vector{\mu}_k$をへの二乗距離の総和を最小にすることで,クラスタリングを行う.

ここで,各データ点$\vector{x}_n$に対し,対応する2値指示変数$r_{nk} \in \{0, 1\}\ (k = 1, \cdots, K)$を定める.
これは,そのデータ点$\vector{x}_n$がクラスタ$k$に割り当てられるかを表す変数である.
すなわち,データ点$\vector{x}_n$がクラスタ$k$に割り当てられる場合は$r_{nk}=1$とし,
$j \not= k$については,$r_{nk}=0$とする.これは,1-of-K符号化法として知られている.

次に,次の目的関数$J$を定義する.

\begin{align}
  J = \sum_{n=1}^{N} \sum_{k=1}^{K} r_{nk} || {\displaystyle \vector{x}_n - \vector{\mu}_k} || ^2
\end{align}

これは,歪み尺度と呼ばれ,各データ店からそれらが割り当てられたベクトル$\vector{\mu}_k$までの二乗距離の総和を表している.
K-meansによるクラスタリングは,$J$を最小にする$\{r_{nk}\}$と$\{\vector{\mu}_k\}$の値を求めることである.

\begin{figure}[htbp]
  \begin{minipage}{0.33\hsize}
    \begin{center}
      \includegraphics[width=40mm]{img/kmeans/Figure91a.pdf}
    \end{center}
  \end{minipage}
  \begin{minipage}{0.33\hsize}
    \begin{center}
      \includegraphics[width=40mm]{img/kmeans/Figure91b.pdf}
    \end{center}
  \end{minipage}
  \begin{minipage}{0.33\hsize}
    \begin{center}
      \includegraphics[width=40mm]{img/kmeans/Figure91c.pdf}
    \end{center}
  \end{minipage}\\
  \begin{minipage}{0.33\hsize}
    \begin{center}
      \includegraphics[width=40mm]{img/kmeans/Figure91d.pdf}
    \end{center}
  \end{minipage}
  \begin{minipage}{0.33\hsize}
    \begin{center}
      \includegraphics[width=40mm]{img/kmeans/Figure91e.pdf}
    \end{center}
  \end{minipage}
  \begin{minipage}{0.33\hsize}
    \begin{center}
      \includegraphics[width=40mm]{img/kmeans/Figure91f.pdf}
    \end{center}
  \end{minipage}\\
  \begin{minipage}{0.33\hsize}
    \begin{center}
      \includegraphics[width=40mm]{img/kmeans/Figure91g.pdf}
    \end{center}
  \end{minipage}
  \begin{minipage}{0.33\hsize}
    \begin{center}
      \includegraphics[width=40mm]{img/kmeans/Figure91h.pdf}
    \end{center}
  \end{minipage}
  \begin{minipage}{0.33\hsize}
    \begin{center}
      \includegraphics[width=40mm]{img/kmeans/Figure91i.pdf}
    \end{center}
  \end{minipage}
  \caption{K-meansの動作}
  \label{fig:k-means}
\end{figure}

K-meansによるクラスタリングは,事前にクラスタ数を指定することによりクラスタリングを行う.
したがって,クラスタ数が未知の場合,K-meansを用いることはできない.


% 手法
\section{手法}
\subsection{Kullback-Leibler情報量}
偶然を伴う現象は,ある確率分布に従う確率変数の実現値であると考えることができる.
この確率分布を近似するモデル(以後「モデル」)はデータを生成する真の確率分布に
どの程度近いかによって評価することができる.
また,データにモデルを当てはめることは,データから真の確率分布を推定しているものと
みなすことができる.このようにモデルと真の分布が共に確率分布であると見なし,
モデルの評価や推定を行う.

真の分布とモデルの近さを図る客観的な規準としてKullback-Leibler情報量(以後「K-L情報量」)がある.
連続型の確率分布のとき,$g(x)$を真の確率密度関数,$f(x)$をモデルが定める確率密度関数とすると,
モデルに関する真の分布のK-L情報量は$\log\{g(X)/f(X)\}$の期待値を取り\eqref{eq:k-l-div}で表される.
\begin{align}
  \label{eq:k-l-div}
  I(g \mid f) &= E_X\left(\log \left\{\frac{g(X)}{f(X)}\right\}\right)\\\nonumber
              &= \int^{\infty}_{-\infty}\log\left\{\frac{g(x)}{f(x)}\right\}g(x)dx
\end{align}
ただし,$\log$は自然対数で,注記がない限り一貫してこの意味で用いる.

このように,真の分布がわかっている場合にはK-L情報量によってモデルの良し悪しを比較できた.
しかし,通常,真の分布は未知で,真の分布から得られたデータだけが与えられていることがい.
したがって,データからK-L情報量を推定する必要がある.
$g(x)$,$f(x)$をそれぞれ真の分布とモデルに対応する密度関数とすると,
\eqref{eq:k-l-div}より,
\begin{align*}
  I(g \mid f) &= \int^{\infty}_{-\infty}\left\{\log\frac{g(x)}{f(x)}\right\}g(x)dx\\\nonumber
              &= -\int^{\infty}_{-\infty}\{\log f(x)\}g(x)dx - 
                 \left(-\int^{\infty}_{-\infty}\{\log g(x)\}g(x)dx\right)
\end{align*}
となるが,右辺の第2項は定数であり,したがって右辺第1項が大きいほどK-L情報量$I(g \mid f)$は
小さくなることがわかる.右辺第1項の$\int^{\infty}_{-\infty}\{\log f(x)\}g(x)dx$は,
確率密度関数$\log f(x)$の期待値$E( \log f(x))$であり,平均対数尤度と呼ばれている.
ここで,
\begin{align*}
  \sum_{i=1}^{n}\log f(x_i)
\end{align*}
を対数尤度と呼ぶことにすると,$n$個の独立な観測値$\{x_1, x_2, \cdots, x_i\}$が得られると,
この平均対数尤度は,対数尤度の$n$分の1
\begin{align*}
  \frac{1}{n}\sum_{i=1}^{n}\log f(x_i)
\end{align*}
で近似される.
したがって,符号に注意すると,対数尤度が大きいほど,そのモデルは真の分布に近いと考えられる.
このようにして,対数尤度をK-L情報量の推定値を考えることにすると異なったタイプのモデルの
良し悪しも比較できるのである.

ところで,確率変数$(X_1, X_2, \cdots, X_n)$の同時密度関数が$f(x_1, x_2, \cdots, x_n \mid \theta)$で
与えられているものとする.
$\theta$は確率密度関数を規定するパラメータである.この時,観測値$(x_1, x_2, \cdots, x_n)$は
与えられたものとして固定し,$f$を$\theta$の関数と考える時,この関数を\textbf{尤度}と呼び,
$L(\theta)$で表す.すなわち,
\begin{align*}
  L(\theta) = f(x_1, x_2, \cdots, x_n \mid \theta)
\end{align*}
である.特に,確率変数が独立な場合には$(X_1, X_2, \cdots, X_n)$の確率密度関数は,
各$X_i (i = 1, \cdots, n)$の確率密度関数の積に等しいことから,
\begin{align*}
  L(\theta) = f(x_1 \mid \theta)f(x_2 \mid \theta) \cdots f(x_n \mid \theta)
\end{align*}
となる.この両辺の対数をとると,すでに求められた対数尤度関数
\begin{align*}
  l(\theta) = \sum_{i=1}^{n}\log f(x_i \mid \theta)
\end{align*}
が導かれる.

ここでは,平均対数尤度の推定量から対数尤度を直接導入した.
しかし,モデルが確率分布の形で与えられている場合には,まず観測値の同時分布から
尤度を定義し,その対数として対数尤度を求めるほうが都合が良い.
$(X_1, X_2, \cdots, X_n)$が独立でない場合にも,尤度の対数として対数尤度
\begin{align*}
  l(\theta) = \log f(x_1, \cdots, x_n \mid \theta)
\end{align*}
が定義できる.

\subsection{最尤法}
ここまで,データに基づいてK-L情報量の大小を比較するためには対数尤度を比較すれば良いことを示した.
あらかじめ与えられたいくつかのモデルがある場合には,対数尤度が最大となるモデルを選択することによって,
近似的には真の分布にいちばん近いモデルが得られることになる.
したがって,モデルがいくつかの調整できるパラメータを保つ場合には,対数尤度を最大とするように
パラメータの値を選ぶことによって良いモデルが得られることがわかる.
この推定を最大尤度法,略して\textbf{最尤法}と呼ばれている.
また,最尤法で導かれた推定量は\textbf{最尤推定量}と呼ばれ,この最尤推定量によって定められるモデルが
最尤モデルである.
最尤モデルの対数尤度を\textbf{最大対数尤度}という.

\subsection{情報量規準}
情報量規準とは,最尤推定によって当てはめられたモデルが複数個あるときに,その中の一つを選択する規準である.
ここでは,最も有名な情報量規準のひとつであるAIC(Akaike Information Criterion; 赤池情報量規準)$^2)$を例に取り
理論の解説を行う.

\subsection{X-means}
先に紹介したK-meansは,クラスタ数を事前に指定する必要が生じる.
しかし,実際にデータのクラスタリングを行う際,クラスタ数が事前に与えられることは少ない.

X-means$^{3)}$は,データ分布が混合等方Gauss分布から生成されたと想定してクラスタ数推定及び
クラスタリングを行う手法である.
K-meansの逐次繰り返しと,BIC$^{4)}$ (Bayesian Information Criterion; ベイズ情報量規準) による
分割停止規準を用いることで,クラスタ数を推定しクラスタリングを実行する.

具体的には以下の手順で行われる.
\begin{enumerate}
    \item クラスタ数$k$を初期化する (通常は$k=2$) .
    \item K-meansを実行する.
    \item 次の処理を$j=1$から$j=k$まで繰り返す.
    \begin{enumerate}
        \item クラスタ$j$のBIC$_j$を計算する.
        \item クラスタ$j$に所属するデータに対し,クラスタ数2としてK-meansを行う.
        \item クラスタ数2としてクラスタリングした結果に対しBIC'$_j$を計算する.
        \item BIC$_j$とBIC'$_j$を比較し,BIC'$_j$が大きければクラスタ数$k$に1を足す.
    \end{enumerate}
    \item 前の処理で$k$が増加した場合は処理2へ戻る.そうでない場合は終了する.
\end{enumerate}

X-meansで用いるBICは次のように求められる.
$d$次元のデータ${\bm D}=({\bm x_0}, {\bm x_1}, \cdots, {\bm x_d})$を
$K$個のクラスタに分割することを考える.
モデル$M_j$の評価に用いるBICは以下で与えられる.
\begin{align}
  \label{eq:bic}
  \mathrm{BIC}(M_j) = \hat{l}_j(D) - \frac{p_j}{2}\ln R
\end{align}
$p_j$はモデル$M_j$のパラメータ数であり,$R$は$M_j$のデータ数,
$\hat{l}_j(D)$は$p$変量Gauss分布の対数尤度関数である.

等方Gauss分布を考えると分散$\sigma^2$は\eqref{eq:variance}により表される.
\begin{align}
  \label{eq:variance}
  \hat{\sigma}^2 = \frac{1}{R-K}\sum_i\left({\bm x}_i-{\bm \mu}_{(i)}\right)^2
\end{align}
すると,確率は次で表される.
\begin{align}
  \label{eq:gaussian-distribution}
  \hat{P}(x_i) = \frac{R_{(i)}}{R}\frac{1}{\sqrt{2\pi}\hat{\sigma}^d}
    \exp\left(-\frac{1}{2\hat{\sigma}^2}||{\bm x}_i-{\bm \mu}_{(i)}||^2\right)
\end{align}
ここで${\bm \mu}_{i}$は$d$次元の平均ベクトルである.
したがって対数尤度関数は
\begin{align}
  \label{eq:log-likelihood}
  l(D) &= \log \prod_i P(x_i) \\\nonumber
  &= \sum_i \left( \log\frac{1}{\sqrt{2\pi}\sigma^d}-\frac{1}{2\sigma^2}||{\bm x}_i-{\bm \mu}_{(i)}||^2 + \log\frac{R_{(i)}}{R} \right)
\end{align}
となる.
ここでクラスタ$n (1 < n < K)$のデータ$D_n$に着目する.
クラスタ$n$のデータ数を$R_n$と表記すると,\eqref{eq:log-likelihood}は以下で表される.
\begin{align}
  \begin{split}
    \hat{l}(D_n) &= -\frac{R_n}{2}\log(2\pi) - \frac{R_n \cdot d}{2}\log(\hat{\sigma}^2) -
    \frac{R_n - K}{2}\\ &
    + R_n\log R_n - R_n \log R
  \end{split}
\end{align}

% 実験結果
\section{クラスタリング実験}

\subsection{実験環境}
実験にはPython3.5を用い,
機械学習のライブラリとしてTensorFlowを用いてアルゴリズムを実装した.

\subsection{精度の評価}
クラスタリング精度の評価はPythonのライブラリであるscilit-learnを用い,以下の3項目により行った.
\begin{description}
  \item[ARI; Adjusted Rand Index, 調整ランド指数]~\\
    クラスタの正解ラベルに対してクラスタリング結果の一致度を評価する指標.1に近づくほどよい結果.
  \item[NMI; Normalized Mutual Information, 正規化相互情報量]~\\
    相互情報量を正規化した尺度.
  \item[Purity]~\\
    生成されたクラスタがどれだけ多数派で占められているかを表す尺度
\end{description}

また,推定されたクラスタ数および,複数回クラスタリングを繰り返し,その際に推定されたクラスタ数の分散も算出した.

\subsection{2次元空間のクラスタリング}

まず,2次元のデータのクラスタリング結果を比較する.
2次元空間に\figurenum{fig:2dim}のように分散$\sigma^2=1$のGauss分布により,
クラスタあたりのサンプル数が500の5つのクラスタを生成し,
対数尤度関数,BIC, AIC, cAICを分割停止規準として採用してクラスタリングを行った.

\tablenum{table:2dim}に都度ランダムに生成されたデータに対して100回クラスタリングを行ったときの,
推定されたクラスタ数,ARI, NMI, Purityの平均値および推定されたクラスタ数の分散を示す.

\begin{table}[htb]
  \centering
  \caption{2次元空間におけるクラスタリング結果}
  \label{table:2dim}
  \begin{tabular}{|c|c|c|c|c|} \hline
    分割停止規準 & クラスタ数(分散) & ARI & NMI & Purity \\\hline
    BIC & 4.58 (0.9836) & 0.84458792 & 0.88281495 & 0.84458792\\
    cAIC & 4.55 (0.6475) & 0.85329139 & 0.89992544 & 0.85329139\\
    AIC & 4.69 (3.8739) & 0.83642236 & 0.88147442 & 0.83642236\\
    対数尤度関数 & 5.32 (10.236) & 0.85699618 & 0.91572100 & 0.85699618\\\hline 
  \end{tabular}
\end{table}

\tablenum{table:2dim}より,2次元空間においてXmeansの分割停止規準としてBICとcBICの間には
クラスタリング結果に大きな差がないことが読み取れる.
クラスタ数の推定もおおよそ適当であり,推定したクラスタ数の分散もあまり大きな値とはなっていない.

また,AICを分割停止規準として採用した場合に着目すると,推定したクラスタ数の分散が非常に大きくなっている.
これは,3.3節で述べたように,AICはパラメータ数を過大に見積もってしまうことに起因するものと思われる.
実際に,推定されたクラスタ数を見ると,クラスタ数を20や22と推定しているものが多く存在する.
しかし,素の対数尤度関数を分割停止規準として採用した場合のクラスタリング結果と比較すると,
比較的安定したクラスタ数推定を行っていることが見て取れる.

\begin{figure}[htbp]
  \begin{center}
    \includegraphics[width=0.7\linewidth]{./img/BIC_2.pdf}
      \caption{2次元空間のクラスタリング例}
      \label{fig:2dim}
  \end{center}
\end{figure}

\subsection{3次元空間のクラスタリング}

3次元空間に\figurenum{fig:3dim}のように分散$\sigma^2=1$のGauss分布により,
クラスタあたりのサンプル数が500の5つのクラスタを生成し,
対数尤度関数,BIC, AIC, cAICを分割停止規準として採用してクラスタリングを行った.

\tablenum{table:3dim}に都度ランダムに生成されたデータに対して100回クラスタリングを行ったときの,
推定されたクラスタ数,ARI, NMI, Purityの平均値および推定されたクラスタ数の分散を示す.

\begin{table}[htb]
  \centering
  \caption{3次元空間におけるクラスタリング結果}
  \label{table:3dim}
  \begin{tabular}{|c|c|c|c|c|} \hline
    分割停止規準 & クラスタ数(分散) & ARI & NMI & Purity \\\hline
    BIC & 4.95 (0.0669) & 0.97179074 & 0.97913818 & 0.97179074\\
    cAIC & 4.92 (0.2313) & 0.96312702 & 0.97023920 & 0.96312702\\
    AIC & 4.88 (0.1443) & 0.95216819 & 0.96855698 & 0.95216819\\
    対数尤度関数 & 5.12 (4.1443) & 0.95731637 & 0.96541468 & 0.95731637\\\hline 
  \end{tabular}
\end{table}

\begin{figure}[htbp]
  \begin{center}
    \includegraphics[width=0.7\linewidth]{./img/BIC_3.pdf}
      \caption{3次元空間のクラスタリング例}
      \label{fig:3dim}
  \end{center}
\end{figure}
2次元のクラスタリングとは異なり,3次元空間においてはBICを分割停止規準として採用した場合の精度が良くなっている.
分散値も非常に小さいため,安定して精度の高いクラスタリングを行っていることが伺える.

cAICとAICの場合を比較した場合,cAICのほうがクラスタリングの精度が高いことがわかる.
しかし,AICにおけるクラスタリングでは,2次元空間ほど推定されたクラスタ数の分散が大きくないことがわかる.
2次元空間では,クラスタ数を過大に見積もってしまう問題があったが,
3次元空間においてその問題は発生していなかった.


% おわりに
\section{おわりに}

本研究では,いくつかの情報量規準を分割停止規準として採用してX-meansでクラスタ数推定を行った.
その結果,2次元空間における混合等方Gauss分布から生成したデータセットのクラスタリングにおいてはBICやcAICが,
3次元空間における混合等方Gauss分布から生成したデータセットのクラスタリングにおいてはBICが適していることがわかった.
2次元空間においてはAICを採用した場合,クラスタ数を過大に見積もってしまう問題が見受けられた.

以上より,混合等方Gauss分布により生成されたデータのクラスタリングを行う際は
BICを分割停止規準として採用することで適切なクラスタリングを行うことができると
言える.

今後は,AIC,cAIC,BIC以外の情報量規準を用いたクラスタリングや,
クラスタリング対象のデータを変更するなどして,それぞれのデータのクラスタリングに
最も適した情報量規準を探求していきたい.


% 参考文献
\begin{thebibliography}{99}
  \bibitem{k-means}
    James MacQueen et al.:
    Some methods for classification and analysis of multivariate observations,
    Proceedings of the fifth Berkeley symposium on mathematical statistics and probability,
    Vol. 1, No. 14, pp. 281--297 (1967).
  \bibitem{mean-shift}
    奥富 正敏他:
    ディジタル画像処理[改訂新版],
    pp.205-208, 公益財団法人 画像情報教育振興協会 (2015).
  \bibitem{x-means}
    Dan Pelleg, Andrew W Moore, et al.:
    X-means: Extending K-means with Efficient Estimation of the Number of Clusters.,
    ICML, Vol. 1, pp. 727--734 (2000).
  \bibitem{AIC}
    Akaike, H.: 
    Information theory and an extension of the maximum likelihood principle, 
    Proceedings of the 2nd International Symposium on Information Theory, 
    pp. 267-281 (1973).
  \bibitem{cAIC}
    Sugiura, N.:
    Further analysts of the data by akaike's information criterion and the finite corrections:
    Further analysts of the data by akaike's, 
    Communications in Statistics-Theory and Methods, 7(1), pp. 13-26 (1978).
  \bibitem{BIC}
    Gideon Schwarz et al.:
    Estimating the dimension of a model.,
    The annals of statistics Vol. 6, No. 2, pp. 461-464 (1978).
\end{thebibliography}


\end{document}
