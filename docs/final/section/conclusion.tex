\section{結論}

本研究では,いくつかの情報量規準を分割停止規準として採用してX-meansでクラスタ数推定を行った.
X-meansは確率分布をベースとしたクラスタリング手法であり,本研究では等方Gauss分布をモデルとして実験を行った.
実データ (MNIST) のクラスタリングを行った場合,等方Gauss分布をモデルとするX-meansによる適切なクラスタリングは
行うことができないことがわかった.
適切なクラスタリングを行うためには,他のモデルを用いるなどの工夫をする必要がある.
確率ベースではないクラスタリング手法としてMean shiftによるクラスタリングも同様に行ったが,
確率密度の極大点を見つけることができず,適切なクラスタリング結果を得ることはできなかった.
また,人工データのクラスタリングの例として,混合等方Gauss分布から生成したデータのクラスタリングを行った.
2次元空間における混合等方Gauss分布から生成したデータセットのクラスタリングにおいてはBICやcAICが,
3次元空間における混合等方Gauss分布から生成したデータセットのクラスタリングにおいてはBICが適していることがわかった.
2次元空間においてはAICを採用した場合,クラスタ数を過大に見積もってしまう問題が見受けられた.
以上より,混合等方Gauss分布により生成されたデータのクラスタリングを行う際は
等方Gauss分布をモデルとするX-meansの分割停止規準としてBICを採用することで適切なクラスタリングを行うことができるといえる.
今後,実データを適切にクラスタリングするための手法を検討していく必要がある.
