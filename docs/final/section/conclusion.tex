\section{結論}

本研究では,いくつかの情報量規準を分割停止規準として採用してX-meansでクラスタ数推定を行った.
その結果,2次元空間における混合等方Gauss分布から生成したデータセットのクラスタリングにおいてはBICやcAICが,
3次元空間における混合等方Gauss分布から生成したデータセットのクラスタリングにおいてはBICが適していることがわかった.
2次元空間においてはAICを採用した場合,クラスタ数を過大に見積もってしまう問題が見受けられた.
以上より,混合等方Gauss分布により生成されたデータのクラスタリングを行う際は
BICを分割停止規準として採用することで適切なクラスタリングを行うことができると言える.

また,実データの例としてMNISTのクラスタリングを行った.
その結果,X-meansを用いる場合はモデルを,Mean shiftを用いる場合は超球の半径などの
パラメータを適切に調整しなければ適切なクラスタリングを行うことができないとわかった.

クラスタリングを行う際には,適切なパラメータを人間が指定する必要があると考えられる.

