\begin{abstract}
Clustering is a one of methods to divide data into several groups called cluster.
It is used in various fields such as data analysis, data mining, image processing, and pattern recognition.
In many clustering methods including $k$-means, it is assumed that the number of clusters is known in advance and clustering is performed by specifying the number of clusters.
In spite of it, in general, there are few data whose number of cluster is known in advance.
So, even when the number of clusters is unknown, it is important to estimate the number of clusters appropriately and perform clustering.
X-means is one way to estimate the number of clusters. 
It is to suppose to the data generated from mixed isotropic Gauss distribution, and estimates the number of clusters by estimating the parameters of the probability distribution.
It uses an estimator called information criterion.
it estimates the number of clusters suppose by the data was generated by mixed isotropic Gauss distribution.
Many of information criterions are proposed by resarchers like AIC (Akaike Information Criterion) and BIC (Bayesian Information Criterion).
In to estimate the number of clusters, however, which it is not known which information criterion is valid for what kind of data.
In this study, the optimal information criterion used for estimating the number of clusters is clarified through quantitative experiments.
\end{abstract}
