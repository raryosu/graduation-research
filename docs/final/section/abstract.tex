\begin{abstract}
Clustering is to divide data into several groups called cluster.
It is used in various fields such as data analysis, data mining, image processing, and pattern recognition.
The $k$-means is the most famous clustering methods.
It is assumed that the number of clusters is known in advance and clustering is performed by specifying the number of clusters.
In spite of it, in general, there are few data whose number of cluster is known in advance.
Even when the number of clusters is unknown, it is important to estimate the number of clusters appropriately.
X-means is one way to estimate the number of clusters. 
It is to suppose to the data generated from mixed isotropic Gaussian distribution, and estimates the number of clusters by estimating the parameters of the probability distribution.
It uses an estimator called information criterion.
Many of information criterions are proposed by resarchers like AIC (Akaike Information Criterion) and BIC (Bayesian Information Criterion).
In to estimate the number of clusters, however, it is not known which information criterion is valid for what kind of data.
In this study, the optimal information criterion used for estimating the number of clusters is clarified through quantitative experiments.
\end{abstract}
